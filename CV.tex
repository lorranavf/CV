%%%%%%%%%%%%%%%%%
% This is an example CV created using altacv.cls (v1.1.5, 1 December 2018) written by
% LianTze Lim (liantze@gmail.com), based on the
% Cv created by BusinessInsider at http://www.businessinsider.my/a-sample-resume-for-marissa-mayer-2016-7/?r=US&IR=T
%
%% It may be distributed and/or modified under the
%% conditions of the LaTeX Project Public License, either version 1.3
%% of this license or (at your option) any later version.
%% The latest version of this license is in
%%    http://www.latex-project.org/lppl.txt
%% and version 1.3 or later is part of all distributions of LaTeX
%% version 2003/12/01 or later.
%%%%%%%%%%%%%%%%

%% If you are using \orcid or academicons
%% icons, make sure you have the academicons
%% option here, and compile with XeLaTeX
%% or LuaLaTeX.
% \documentclass[10pt,a4paper,academicons]{altacv}

%% Use the "normalphoto" option if you want a normal photo instead of cropped to a circle
% \documentclass[10pt,a4paper,normalphoto]{altacv}

\documentclass[10pt,a4paper,ragged2e]{altacv}

%% AltaCV uses the fontawesome and academicon fonts
%% and packages.
%% See texdoc.net/pkg/fontawecome and http://texdoc.net/pkg/academicons for full list of symbols. You MUST compile with XeLaTeX or LuaLaTeX if you want to use academicons.

% Change the page layout if you need to
\geometry{left=1cm,right=9cm,marginparwidth=6.8cm,marginparsep=1.2cm,top=1.25cm,bottom=1.25cm}

% Change the font if you want to, depending on whether
% you're using pdflatex or xelatex/lualatex
\ifxetexorluatex
  % If using xelatex or lualatex:
  \setmainfont{Carlito}
\else
  % If using pdflatex:
  \usepackage[utf8]{inputenc}
  \usepackage[T1]{fontenc}
  \usepackage[default]{lato}
\fi

% Change the colours if you want to
\definecolor{VividPurple}{HTML}{3E0097}
\definecolor{SlateGrey}{HTML}{2E2E2E}
\definecolor{LightGrey}{HTML}{37474F}
\colorlet{heading}{VividPurple}
\colorlet{accent}{VividPurple}
\colorlet{emphasis}{SlateGrey}
\colorlet{body}{LightGrey}

% Change the bullets for itemize and rating marker
% for \cvskill if you want to
\renewcommand{\itemmarker}{{\small\textbullet}}
\renewcommand{\ratingmarker}{\faCircle}

%% sample.bib contains your publications
% \addbibresource{sample.bib}

\begin{document}
\name{Lorrana Verdi Flores}
\tagline{Bioinformata}
% Cropped to square from https://en.wikipedia.org/wiki/Marissa_Mayer#/media/File:Marissa_Mayer_May_2014_(cropped).jpg, CC-BY 2.0
\photo{2.5cm}{perfil}
\personalinfo{%
  % Not all of these are required!
  % You can add your own with \printinfo{symbol}{detail}
    \email{lorrana.vf@gmail.com}
    % \phone{(35) 99827-3814}
    \location{Lavras, MG}
    \linkedin{linkedin.com/in/lorranaflores}
    \github{https://lorranavf.github.io/}
%   \orcid{orcid.org/0000-0000-0000-0000} % Obviously making this up too. If you want to use this field (and also other academicons symbols), add "academicons" option to \documentclass{altacv}
}

%% Make the header extend all the way to the right, if you want.
\begin{fullwidth}
\makecvheader
\end{fullwidth}

%% Depending on your tastes, you may want to make fonts of itemize environments slightly smaller
\AtBeginEnvironment{itemize}{\small}

%% Provide the file name containing the sidebar contents as an optional parameter to \cvsection.
%% You can always just use \marginpar{...} if you do
%% not need to align the top of the contents to any
%% \cvsection title in the "main" bar.

\cvsection[page1sidebar]{Experiências Profissionais}

\cvevent{UFLA}{Pesquisadora}{Abril 2021 -- Atual}{Lavras, MG}
\begin{itemize}
    \item Desenvolvimento de uma Plataforma de Descoberta de Genes de interesse agronômicos utilizando técnicas de Machine Learning e Integração de Dados Genômicos.
    \item Desenvolvimento de uma Plataforma de Apoio à Descoberta de Genes Pesticidas em um Programa de Prospecção de cepas da bactéria \textit{Bacillus thuringiensis}.
    \item Desenvolvimento de Pipelines de Identificação e Caracterização de Genes.
    \item Prospecção de transcritos codificadores de enzimas envolvidas no metabolismo de quitina no transcritoma da formiga cortadeira Atta sexdens.
    \item Montagem do transcritoma da formiga cortadeira \textit{Atta sexdens}.
    \item Análise de expressão gênica de bibliotecas públicas de RNA-Seq de \textit{Solanum lycopersicum}.
    \item Prospecção de genes candidatos (WUSCHEL e BBM) em \textit{Eucalyptus grandis}.
\end{itemize}

\vspace{10px}

\cvevent{Instituto Mato-Grossense do Algodão}{Téccnico de Laboratório}{Setembro 2016-- Janeiro 2021}{Rondonópolis, MT}
\begin{itemize}
    \item Gerenciamento do Laboratório de Transformação Genética.
    \item Treinamento de pessoas.
    \item Controle de demandas de manutenção.
    \item Controle de estoques e pedidos.
    \item Planejamento e execução de experimentos de transformação genética de algodão. \\
\end{itemize}

\vspace{10px}

\cvsection{Educação}

\cvevent{Universidade Federal de Lavras}{Doutorado em Biotecnologia Vegetal}{Fevereiro 2023 -- Atual}{Lavras, MG}
\begin{itemize} 
    \item Trabalho em andamento: Desenvolvimento de uma plataforma de descobrimento de Genes.
    \item Linhas de Pesquisa: Genômica regulatória, Integração multi-ômica, Desenvolvimento de modelos preditivos.
    \item Área de atuação: Bioinformática. 
\end{itemize} 

\vspace{10px}

\cvevent{Universidade Federal de Lavras}{Mestrado em Biotecnologia Vegetal}{Abril 2021 -- Jan 2023}{Lavras, MG}
\begin{itemize} 
    \item Trabalho final: Prospecção de transcritos codificadores de quitinases em dados de RNA-seq da formiga cortadeira \textit{Atta sexdens}.
    \item Linha de Pesquisa: Análise Genômica e Funcional.
    \item Área de atuação: Bioinformática. 
\end{itemize} 

\clearpage

\end{document}